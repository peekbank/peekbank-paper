% Template for Cogsci submission with R Markdown

% Stuff changed from original Markdown PLOS Template
\documentclass[10pt, letterpaper]{article}

\usepackage{cogsci}
\usepackage{pslatex}
\usepackage{float}
\usepackage{caption}

% amsmath package, useful for mathematical formulas
\usepackage{amsmath}

% amssymb package, useful for mathematical symbols
\usepackage{amssymb}

% hyperref package, useful for hyperlinks
\usepackage{hyperref}

% graphicx package, useful for including eps and pdf graphics
% include graphics with the command \includegraphics
\usepackage{graphicx}

% Sweave(-like)
\usepackage{fancyvrb}
\DefineVerbatimEnvironment{Sinput}{Verbatim}{fontshape=sl}
\DefineVerbatimEnvironment{Soutput}{Verbatim}{}
\DefineVerbatimEnvironment{Scode}{Verbatim}{fontshape=sl}
\newenvironment{Schunk}{}{}
\DefineVerbatimEnvironment{Code}{Verbatim}{}
\DefineVerbatimEnvironment{CodeInput}{Verbatim}{fontshape=sl}
\DefineVerbatimEnvironment{CodeOutput}{Verbatim}{}
\newenvironment{CodeChunk}{}{}

% cite package, to clean up citations in the main text. Do not remove.
\usepackage{apacite}

% KM added 1/4/18 to allow control of blind submission


\usepackage{color}

% Use doublespacing - comment out for single spacing
%\usepackage{setspace}
%\doublespacing


% % Text layout
% \topmargin 0.0cm
% \oddsidemargin 0.5cm
% \evensidemargin 0.5cm
% \textwidth 16cm
% \textheight 21cm

\title{Peekbank: Exploring child lexical processing through a large-scale
open-source database of developmental eyetracking datasets}


\author{{\large \bf Martin Zettersten (martincz@princeton.edu)} \\ Department of Psychology, South Dr \\ Princeton, NJ 08540 USA \AND {\large \bf CLinger Xu (txu@iu.edu)}  \AND {\large \bf Stephan Meylan (smeylan@mit.edu)}  \AND {\large \bf Mika Braginsky (mikabr@mit.edu)}  \AND {\large \bf George Kachergis (kachergis@stanford.edu)}  \AND {\large \bf Molly Lewis (mollyllewis@gmail.com)}  \AND {\large \bf Claire Bergey (cbergey@uchicago.edu)}  \AND {\large \bf Naiti S. Bhatt (nbhatt@hmc.edu)}  \AND {\large \bf Veronica Boyce (vboyce@stanford.edu)}  \AND {\large \bf Jessica Mankewitz (jmankewitz@stanford.edu)} \AND {\large \bf Bria Long (bria@stanford.edu)} 
 \AND {\large \bf Daniel Yurovsky (yurovsky@stanford.edu)} 
\AND {\large \bf Annissa Saleh (ans638@nyu.edu)}  \AND {\large \bf Sarp Uner (sarp.uner@duke.edu)}  \AND {\large \bf Alexandra Carstensen (abcarstensen@stanford.edu)}  \AND {\large \bf Angeline Sin Mei Tsui (astsui@stanford.edu)}   \AND {\large \bf CMichael C. Frank (mcfrank@stanford.edu)}}


\begin{document}

\maketitle

\begin{abstract}
Lexical processing skills -- the ability to rapidly process words and
link them to referents in context -- are central to children's early
language development. Children's lexical processing is typically studied
in the looking-while-listening paradigm, which measures infants'
fixation of a target object (vs.~a distracter) after hearing a target
label. We present a large-scale open-source database of infant and
toddler eye-tracking data from looking-hile-listening tasks. The goal of
this database is to address theoretical and methodological challenges in
measuring infant vocabulary development. We present two analyses of the
current database (N=1,233): (1) models capturing age-related changes in
infants' lexical processing while generalizing across item-level
variability and (2) an analysis of how a central methodological decision
-- selecting the time window of analysis -- impacts measure reliability.
Future efforts will expand the scope of the current database to advance
our understanding of participant-level and item-level variation in
children's vocabulary development.

\textbf{Keywords:}
lexical processing; eyetracking; database; vocabulary development;
looking-while-listening
\end{abstract}

\hypertarget{introduction}{%
\section{Introduction}\label{introduction}}

Across their first years of life, children learn words in their native
tongues at a rapid pace (Frank, Braginsky, Yurovsky, \& Marchman, 2021).
A key part of the word learning process is children's ability to rapidly
process words and link them to relevant meanings in context - often
referred to as lexical processing. Developing lexical processing skills
builds a foundation for children's language development and is
predictive of both linguistic and more general cognitive outcomes later
in life (Bleses, Makransky, Dale, Højen, \& Ari, 2016; Marchman et al.,
2018).

Lexical processing is traditionally studied in
``looking-while-listening'' studies (alternatively referred to as the
intermodal preferential looking procedure) (Fernald, Zangl, Portillo, \&
Marchman, 2008; Hirsh-Pasek, Cauley, Golinkoff, \& Gordon, 1987). In
such studies, infants listen to a sentence prompting a specific referent
(e.g., \emph{Look at the dog!}) while viewing two images on the screen
(e.g., an image of a dog - the target image - and an image of a duck -
the distractor image). Infants' lexical processing is measured in terms
of how quickly and accurately infants subsequently fixate the correct
target image after hearing its label. Studies using this basic design
have contributed to our understanding of a wide range of questions in
language development (Golinkoff, Ma, Song, \& Hirsh-Pasek, 2013),
including infants' early noun knowledge (Bergelson \& Swingley, 2012),
phonological representations of words ({\textbf{???}}), prediction
during language processing (Lew-Williams \& Fernald, 2007), and
individual differences in language development (Marchman et al., 2018).

While the looking-while-listening paradigm has been highly fruitful in
advancing understanding of early word knowledge, fundamental questions
remain both about the trajectory of children's lexical processing
ability and the nature of the method itself. One central question
relates to teasing apart variability due to participants and variability
due to specific items across development. Drawing inferences about
item-level variability is key to many questions in how word learning
unfolds, including how properties of the language input influence
lexical development (Goodman, Dale, \& Li, 2008; Roy, Frank, Decamp,
Miller, \& Roy, 2015). Most studies of infant lexical processing focus
on generalizing performance across participants, and are constrained in
their ability to provide generalizations across the item level - the
level of specific words. Generalizing behavior on the level of both
participants and items simultaneously is often difficult in the context
of a solitary study, especially given practical constraints on the
number of trials (and consequently items) tested within a given infant.
One key to meeting this challenge is having sufficiently large datasets
to account for and explain variability in lexical processing on the item
level.

A second question relates to evaluating methodological best-practices.
In particular, many fundamental analytic decisions vary substantially
across studies. For example, researchers vary in their decisions
regarding how to select time windows for analysis, modeling how lexical
processing unfolds over time, and the appropriate transformations to
perform on the dependent measure of target fixations (Csibra, Hernik,
Mascaro, Tatone, \& Lengyel, 2016; Fernald et al., 2008; Huang \&
Snedeker, 2020). This problem is made more complex by the fact that many
of these decisions likely depend on a variety of design-related and
participant-related factors (e.g., infant age). Establishing best
practices regarding analytic decisions therefore requires a large
database of infant lexical processing studies varying across such
factors, in order to independently test the potential consequences of a
variety of methodological decisions and study factors on the
interpretation of study results.

\hypertarget{peekbank-a-large-scale-database-of-looking-while-listening-studies}{%
\subsection{Peekbank: A large-scale database of
looking-while-listening-studies}\label{peekbank-a-large-scale-database-of-looking-while-listening-studies}}

What these questions and challenges share is that they are difficult to
answer at the scale of a single looking-while-listening study. In order
to address these questions, we introduce \emph{peekbank}, a flexible and
reproducible interface to an open database of developmental eye-tracking
studies. Here, we give a brief overview over the key components of the
peekbank project and some initial demonstrations of its utility in
advancing theoretical and methodological questions in the study of
children's lexical processing. The peekbank project (a) collects a large
set of eye-tracking datasets on children's lexical processing, (b)
introduces a data format and processing tools for standardizing
eyetracking data across different data sources, and (c) provides an API
for quickly accessing and analyzing the database. We report two analyses
using the database and associated tools (N=1,233 unique participants):
(1) a growth-curve analysis modeling age-related changes in infants'
lexical processing while generalizing across item-level variability and
(2) a multiverse-style analysis of how a central methodological decision
-- selecting the time window of analysis -- impacts reliability when
measuring infants' proportion looking to the target.

\hypertarget{methods}{%
\section{Methods}\label{methods}}

\hypertarget{database-framework}{%
\subsection{Database Framework}\label{database-framework}}

\begin{CodeChunk}
\begin{figure}[tb]

{\centering \includegraphics{figs/fig_framework_overview-1} 

}

\caption[Overview of the peekbank data ecosystem]{Overview of the peekbank data ecosystem.}\label{fig:fig_framework_overview}
\end{figure}
\end{CodeChunk}

The Peekbank data framework consists of three libraries that help to
populate and query a relational database (Fig.
\ref{fig:fig_framework_overview}). The \texttt{peekds} library (for the
R language) helps researchers convert and validate existing datasets to
use the relational format used by the database. The \texttt{peekbank}
library (Python) creates a database with the relational schema and
populates it with the standardized datasets produced by \texttt{peekds}.
The database is implemented in MySQL, an industry standard relational
database, which may be accessed by a variety of programming languages
over the internet. The \texttt{peekbankr} library (R) provides an
application programming interface, or API, that provides high-level
abstractions to help researchers run common analysis tasks on the
database.

\hypertarget{data-format-and-processing}{%
\subsection{Data Format and
Processing}\label{data-format-and-processing}}

One of the main challenges in compiling a large-scale eyetracking
dataset is the lack of a shared re-usable data format among labs
conducting individual experiments. Eyetracking methods and researcher
teams vary in their conventions for exporting and structuring data,
rendering the task of integrating datasets from different labs and data
sources difficult. We developed a common, tidy format for the
eyetracking data in peekbank to ease the process of conducting
cross-dataset analyses. The schema of the database is sufficiently
general to handle heterogeneous datasets from many studies from many
labs, including both manually coded and automated eyetracking data.

During data import, raw eyetracking datasets are processed to conform to
the peekbank data schema. The centerpiece of the schema is the
aoi\_timepoints table , which records whether participants looked to the
target or distracter stimulus at each timepoint of a given trial.
Additional tables track information about data sources (datasets),
participant characteristics (subjects, administrations), trial
characteristics (trials, trial\_types), stimuli (stimuli), and raw
eyetracking data information (xy\_timepoints, aoi\_region\_sets). In
addition to unifying the data format, we conduct several additional
pre-processing steps to facilitate analyses across datasets, including
resampling observations to a common sampling rate (40 Hz) and
normalizing time relative to the onset of the target label.

\hypertarget{current-data-sources}{%
\subsection{Current Data Sources}\label{current-data-sources}}

\begin{table}[H]
\centering
\begingroup\fontsize{9pt}{10pt}\selectfont
\begin{tabular}{lrrl}
  \hline
Dataset Name & N & Mean Age & Method \\ 
  \hline
canine & 36 & 23.8 & manual coding \\ 
  coartic & 29 & 20.8 & eyetracking \\ 
  cowpig & 45 & 20.5 & manual coding \\ 
  ft\_pt & 69 & 17.1 & manual coding \\ 
  reflook\_socword & 435 & 33.6 & eyetracking \\ 
  reflook\_v4 & 347 & 37.2 & eyetracking \\ 
  salientme & 44 & 40.1 & manual coding \\ 
  switchingCues & 60 & 44.3 & manual coding \\ 
  tablet & 110 & 33.8 & eyetracking \\ 
  tseltal & 23 & 31.3 & manual coding \\ 
  yoursmy & 35 & 14.5 & eyetracking \\ 
   \hline
\end{tabular}
\endgroup
\caption{Overview over the datasets in the current database.} 
\end{table}

The database currently includes 11 datasets comprising N=1233 total
participants, with 23 to 435 participants per dataset (Table 1). The
vast majority of datasets (10 out of 11 total) consist of monolingual
native English speakers.\\
The datasets span a wide age spectrum with participants ranging from 8
to 84 months of age, and are balanced in terms of gender (48\% female).
The studies in the current database vary across a number of dimensions
related to design and methodology. The database includes studies using
both manually coded video recordings or aumotated eyetracking methods to
measure children's gaze behavior. Most studies focused on testing
familiar items, but the database also includes studies in which both
familiar words and novel pseudowords were tested.

\hypertarget{results}{%
\section{Results}\label{results}}

\hypertarget{general-descriptives}{%
\subsection{General Descriptives}\label{general-descriptives}}

\begin{table}[H]
\centering
\begingroup\fontsize{9pt}{10pt}\selectfont
\begin{tabular}{lrrl}
  \hline
Dataset Name & Unique Items & Prop. Target & 95\% CI \\ 
  \hline
canine & 16 & 0.64 & [0.61, 0.67] \\ 
  coartic & 10 & 0.70 & [0.67, 0.73] \\ 
  cowpig & 12 & 0.60 & [0.58, 0.63] \\ 
  ft\_pt & 8 & 0.64 & [0.63, 0.66] \\ 
  reflook\_socword & 6 & 0.61 & [0.6, 0.63] \\ 
  reflook\_v4 & 10 & 0.63 & [0.61, 0.64] \\ 
  salientme & 16 & 0.73 & [0.71, 0.75] \\ 
  switchingCues & 40 & 0.77 & [0.75, 0.79] \\ 
  tablet & 24 & 0.53 & [0.51, 0.56] \\ 
  tseltal & 30 & 0.59 & [0.54, 0.63] \\ 
  yoursmy & 87 & 0.60 & [0.55, 0.64] \\ 
   \hline
\end{tabular}
\endgroup
\caption{Average Proportion Target Looking in each dataset.} 
\end{table}

In general, participants demonstrated robust, above-chance word
recognition in each dataset. Table 2 shows the average proportion of
target looking within a standard critical window of 300 - 2000ms after
the onset of the label for each dataset. As can be seen in Figure
\ref{fig:peekbank_item_vis}, the number of unique target labels and
there is substantial variability in the accuracy curves associated with
different items across datasets. Proportion target looking was generally
higher for familiar target labels (M = 67.5\%, 95\% CI = {[}66.6\%,
68.5\%{]}) than for novel target labels that participants learned during
the experiment (M = 55.1\%, 95\% CI = {[}53.8\%, 56.3\%{]}).

\begin{CodeChunk}
\begin{figure*}[h]

{\centering \includegraphics{figs/peekbank_item_vis-1} 

}

\caption[Item-level variability in proportion target looking within each dataset]{Item-level variability in proportion target looking within each dataset. Colored lines represent specific target labels.}\label{fig:peekbank_item_vis}
\end{figure*}
\end{CodeChunk}

\hypertarget{predicting-age-related-changes-while-generalizing-across-items}{%
\subsection{Predicting Age-Related Changes While Generalizing Across
Items}\label{predicting-age-related-changes-while-generalizing-across-items}}

Developmental changes in word recognition have been a central issue
since early investigations of eye-tracking techniques (Fernald, Pinto,
Swingley, Weinberg, \& McRoberts, 1998). Children's speed and accuracy
of word recognition increases across early childhood, yet measuring
these increases creates a tricky issue. Words that are appropriate for
an 18-month-old will be far too easy for a three-year-old; those that
are appropriate for a three-year-old will be hard or even unfamiliar to
the 18-month-old. Failure to choose appropriate test items can even lead
to spurious conclusions about development (Peter et al., 2019).

This issue is familiar in psychometrics: test developers interested in
measuring across a wide range of a particular latent ability must choose
items appropriate for different abilities. One solution is to use data
from a bank of questions that have been taken by test-takers of a range
of abilities, and then use item-response theory models to create
different test versions appropriate for different ability ranges
(Embretson \& Reise, 2013). Such tests can then be used to extract
estimates of developmental change that are independent of individual
tests and their particular items.

Peekbank provides the appropriate data for estimating these
item-independent developmental changes and designing age-appropriate
tests in the future. Here we show a proof of concept by providing an
estimate of the item-independent growth of word recognition accuracy
across development. We take advantage of the equivalence between item
response theory and linear mixed effects models (LMMs; De Boeck et al.,
2011), using LMMs to model the trajectory of word recognition across
age. We follow the approach of Mirman (2014) and use growth curve LMMs
to predict the timecourse of recognition. Specifically, we predicted
children's proportion of target looking during an early window of time
(0 - 1500ms), using an empirical logit transform on the proportion of
target looking to allow the use of linear (rather than logistic)
regression models. Our predictors were time after word onset and age,
and we additionally included polynomial functions of time (up to fourth
order) and quadratic effects of age, as well as their interactions. We
subtracted all intercepts to force fits to start at a baseline of 0
(chance performance) at time zero. As a random effect structure, we
included by-item, by-subject, and by-dataset random intercepts; though a
larger random effect structure could be justified by the data, the size
of the dataset precluded fitting these.

Figure \ref{fig:age_gca} depicts the results of this analysis. Panel A
shows the mean empirical word recognition curves for four age groups,
along with fitted model performance. Although model fits are acceptable,
developmental change appears irregular - for example, 12--24 month-olds
show if anything slightly earlier recognition than 24--36 month-olds.
This pattern is an artifact of averaging across datasets with
substantially different items and structures. Panel B shows model
predictions for the population level of each random effect, giving our
best estimates of the latent ability structure. Here we see continuous
increases in both speed (point at which the curve rises) and accuracy
(asymptote of the curve) across ages, though this developmental trend
decelerates (consistent with other work on the development of reaction
times in word recognition and more generally; Frank, Lewis, \&
MacDonald, 2016; Kail, 1991). This proof of concept suggests that
Peekbank can be used to model developmental change over multiple years,
overcoming the limitations of individual datasets.

\begin{CodeChunk}
\begin{figure*}[h]

{\centering \includegraphics{figs/age_gca-1} 

}

\caption[Growth curve models of proportion target looking during the critical target window at each age range (in months)]{Growth curve models of proportion target looking during the critical target window at each age range (in months).}\label{fig:age_gca}
\end{figure*}
\end{CodeChunk}

\hypertarget{time-window-selection}{%
\subsection{Time Window Selection}\label{time-window-selection}}

Taking a similar approach to that of Peelle \& Van Engen (2020), we
conducted a multiverse-style analysis considering possible time windows
researchers might select for analyzing their data. Our multiverse
analysis focuses on the reliability of participants' response to
familiar words by measuring the subject-level inter-item correlation
(IIC) for proportion of looking at familiar targets. The time windows
selected by researchers varies substantially in the literature, with
some studies analyzing shorter time windows between 300 ms and 1800-2000
ms post-target onset ({\textbf{???}}; Fernald et al., 2008), while
longer time windows extending to approximately 3000-4000ms post-target
onset are often also used ({\textbf{???}}; {\textbf{???}}). We thus
examined a broad range of window start and end times: we calculated
subjects' mean IIC across from 300 ms pre-target onset to 1500 ms
post-target word onset) and window end times (ranging from 0 ms to 4000
ms). While it is an open question what space of possible windows will
yield the greatest reliability, we expect to see very low reliability
(i.e.~0) in windows that start and end before target onset, and likely
for any windows that end within 300 ms post-target onset, before
participants have had a chance to execute a response. Since observations
were unevenly distributed across the age range, and because children
likely show a varying response to familiar items as they age, we split
our data into four age bins (12-24, 24-36, 36-48, and 48-60 months). For
each combination of window start time and end time with a minimum window
duration of 50 ms, participants' average inter-item correlation for
proportion of looking at familiar targets was calculated.

The resulting correlations of this multiverse analysis are shown in
Figure \ref{fig:time_window}, where subjects' mean ICC for proportion of
looking to familiar targets for each combination of window start time
and end time is shown as a colored pixel. The analysis shows that ICC is
positive (red) under a wide range of window choices, and generally only
negative (blue) or 0 (white) when both the start time is less than
\textasciitilde500 ms and the end time is less than \textasciitilde1000
ms -- a window choice researchers rarely consider for analyzing word
recognition. Intriguingly, late end times and long overall window
lengths--generally not used by researchers--show the greatest
reliability, suggesting that researchers may consider keeping this data
for analysis rather than discarding it. Moreover, there is some
variation by age group in where the strongest ICCs are found (and in the
overall strength of ICCs).

What general recommendations follow from this analysis? We minimally
consider which start times and window lengths result in IICs of at least
.01, noting that this is still a rather low target. A window length of
at least 1500 ms eliminated 94\% of low ICCs, and this threshold
combined with a start time of at least 300 ms eliminated all but 0.5\%
of low ICCs. A start time of 500 ms and a window length of at least 1500
ms resulted in no IICs \textless{} .01, and an average IIC of 0.08.
However, the overall strength of the IICs are generally much weaker than
might be desired (median = .05), with maximum values of 0.15 (reached
only in 3-year-olds).

\begin{CodeChunk}
\begin{figure*}[h]

{\centering \includegraphics{figs/time_window-1} 

}

\caption[Participants' average inter-item correlation for proportion of looking time to familiar targets, as a function of window start time and end time, with each facet showing a different age group]{Participants' average inter-item correlation for proportion of looking time to familiar targets, as a function of window start time and end time, with each facet showing a different age group. More positive (red) correlations are more desirable, and blue/white represent start/end time combinations that researchers should avoid.}\label{fig:time_window}
\end{figure*}
\end{CodeChunk}

\hypertarget{discussion}{%
\section{Discussion}\label{discussion}}

Many central questions in developmental science face a fundamental data
collection challenge: Studying effects of interest requires a large
amount of observations, but collecting infant data is difficult,
time-intensive, and often limited to a small number of observations per
participant. Recent years have seen a growing effort to build open
source tools and pooling research efforts to meet the challenge of data
collection and aggregation in developmental science (Bergmann et al.,
2018; The ManyBabies Consortium, 2020). Peekbank expands on these
efforts by building an infrastructure for aggregating eyetracking data
across studies, with a particular focus on the looking-while-listening
paradigm. This paper presents a preliminary illustration of some of the
key theoretical and methodological questions the peekbank database aims
to address: generalizing across item-level variability in children's
lexical processing and providing data-driven guidance on methodological
choices.

Diving into more specifics

There are a number of critical limitations surrounding the current scope
of the database. A key priority in future work will be to expand the
size of the database. With 11 datasets currently available in the
dataset, idiosyncrasies of particular designs and condition
manipulations still have substantial influence on modeling results.
Expanding the set of distinct datasets will allow us to increase the
number of observations per item across datasets, allowing for more
robust generalizations regarding participant- and item-level
variability. The current database is also limited by the relatively
homogeneous background of its participants, both with respect to
language (almost entirely monolingual native English speakers) and
cultural background (with one exception, all participants are growing up
in WEIRD environments;). Increasing the diversity of the participant
backgrounds and languages contributing to the current database will
increase the scope of the generalizations we can form about child
lexical processing. Finally, while the current database is current
mainly focused on studies of lexical processing, the tools and
infrastructure developed in the current project can in principle be
expanded to accommodate any eyetracking paradigm used with infants and
toddlers, opening up new avenues for insights into infant development.
Infant looking has been at the core of many of the key advances in our
understanding infant cognition. Aggregating large datasets of infant
looking behavior in a single, openly accessible format promises to bring
a fuller picture of infant cognitive development into view.

\hypertarget{acknowledgements}{%
\section{Acknowledgements}\label{acknowledgements}}

We would like to thank the labs and researchers that have made their
data publicly available in the database.

\hypertarget{references}{%
\section{References}\label{references}}

\setlength{\parindent}{-0.1in} 
\setlength{\leftskip}{0.125in}

\noindent

\hypertarget{refs}{}
\leavevmode\hypertarget{ref-Bergelson2012a}{}%
Bergelson, E., \& Swingley, D. (2012). At 6-9 months, human infants know
the meanings of many common nouns. \emph{Proceedings of the National
Academy of Sciences of the United States of America}, \emph{109}(9),
3253--8. \url{http://doi.org/10.1073/pnas.1113380109}

\leavevmode\hypertarget{ref-Bergmann2018}{}%
Bergmann, C., Tsuji, S., Piccinini, P. E., Lewis, M. L., Braginsky, M.,
Frank, M. C., \& Cristia, A. (2018). Promoting replicability in
developmental research through meta-analyses: Insights from language
acquisition research. \emph{Child Development}, \emph{89}(6),
1996--2009. \url{http://doi.org/10.1111/cdev.13079}

\leavevmode\hypertarget{ref-Bleses2016}{}%
Bleses, D., Makransky, G., Dale, P. S., Højen, A., \& Ari, B. A. (2016).
Early productive vocabulary predicts academic achievement 10 years
later. \emph{Applied Psycholinguistics}, \emph{37}(6), 1461--1476.
\url{http://doi.org/10.1017/S0142716416000060}

\leavevmode\hypertarget{ref-Csibra2016}{}%
Csibra, G., Hernik, M., Mascaro, O., Tatone, D., \& Lengyel, M. (2016).
Statistical treatment of looking-time data. \emph{Developmental
Psychology}, \emph{52}(4), 521--536.
\url{http://doi.org/10.1037/dev0000083}

\leavevmode\hypertarget{ref-de-boeck2011}{}%
De Boeck, P., Bakker, M., Zwitser, R., Nivard, M., Hofman, A.,
Tuerlinckx, F., \ldots{} others. (2011). The estimation of item response
models with the lmer function from the lme4 package in r. \emph{Journal
of Statistical Software}, \emph{39}(12), 1--28.

\leavevmode\hypertarget{ref-embretson2013}{}%
Embretson, S. E., \& Reise, S. P. (2013). \emph{Item response theory}.
Psychology Press.

\leavevmode\hypertarget{ref-fernald1998}{}%
Fernald, A., Pinto, J. P., Swingley, D., Weinberg, A., \& McRoberts, G.
W. (1998). Rapid gains in speed of verbal processing by infants in the
2nd year. \emph{Psychological Science}, \emph{9}(3), 228--231.

\leavevmode\hypertarget{ref-Fernald2008}{}%
Fernald, A., Zangl, R., Portillo, A. L., \& Marchman, V. A. (2008).
Looking while listening: Using eye movements to monitor spoken language
comprehension by infants and young children. In I. A. Sekerina, E. M.
Fernandez, \& H. Clahsen (Eds.), \emph{Developmental psycholinguistics:
On-line methods in children's language processing} (pp. 97--135).
Amsterdam: John Benjamins.

\leavevmode\hypertarget{ref-frank2021}{}%
Frank, M. C., Braginsky, M., Yurovsky, D., \& Marchman, V. A. (2021).
\emph{Variability and Consistency in Early Language Learning: The
Wordbank Project}. Cambridge, MA: MIT Press. Retrieved from
\url{http://wordbank-book.stanford.edu}

\leavevmode\hypertarget{ref-frank2016b}{}%
Frank, M. C., Lewis, M., \& MacDonald, K. (2016). A performance model
for early word learning. In \emph{CogSci}.

\leavevmode\hypertarget{ref-Golinkoff2013}{}%
Golinkoff, R. M., Ma, W., Song, L., \& Hirsh-Pasek, K. (2013).
Twenty-five years using the intermodal preferential looking paradigm to
study language acquisition: What have we learned? \emph{Perspectives on
Psychological Science}, \emph{8}(3), 316--339.
\url{http://doi.org/10.1177/1745691613484936}

\leavevmode\hypertarget{ref-Goodman2008}{}%
Goodman, J. C., Dale, P. S., \& Li, P. (2008). Does frequency count?
Parental input and the acquisition of vocabulary. \emph{Journal of Child
Language}, \emph{35}(3), 515--531.
\url{http://doi.org/10.1017/S0305000907008641}

\leavevmode\hypertarget{ref-Hirsh-Pasek1987}{}%
Hirsh-Pasek, K., Cauley, K. M., Golinkoff, R. M., \& Gordon, L. (1987).
The eyes have it: Lexical and syntactic comprehension in a new paradigm.
\emph{Journal of Child Language}, \emph{14}(1), 23--45.
\url{http://doi.org/10.1017/S030500090001271X}

\leavevmode\hypertarget{ref-Huang2020}{}%
Huang, Y., \& Snedeker, J. (2020). Evidence from the visual world
paradigm raises questions about unaccusativity and growth curve
analyses. \emph{Cognition}, \emph{200}, 104251.
\url{http://doi.org/10.1016/j.cognition.2020.104251}

\leavevmode\hypertarget{ref-kail1991}{}%
Kail, R. (1991). Developmental change in speed of processing during
childhood and adolescence. \emph{Psychological Bulletin}, \emph{109}(3),
490.

\leavevmode\hypertarget{ref-Lew-Williams2007}{}%
Lew-Williams, C., \& Fernald, A. (2007). Young children learning Spanish
make rapid use of grammatical gender in spoken word recognition.
\emph{Psychological Science}, \emph{18}(3), 193--198.
\url{http://doi.org/10.1111/j.1467-9280.2007.01871.x}

\leavevmode\hypertarget{ref-Marchman2018}{}%
Marchman, V. A., Loi, E. C., Adams, K. A., Ashland, M., Fernald, A., \&
Feldman, H. M. (2018). Speed of language comprehension at 18 months old
predicts school-relevant outcomes at 54 months old in children born
preterm. \emph{Journal of Developmental \& Behavioral Pediatrics}, 1.
\url{http://doi.org/10.1097/DBP.0000000000000541}

\leavevmode\hypertarget{ref-Mirman2014}{}%
Mirman, D. (2014). \emph{Growth curve analysis and visualization using
R}. CRC Press.

\leavevmode\hypertarget{ref-Peelle2020}{}%
Peelle, J. E., \& Van Engen, K. J. (2020). Time stand still: Effects of
temporal window selection on eye tracking analysis.
\url{http://doi.org/https://doi.org/10.31234/osf.io/pc3da}

\leavevmode\hypertarget{ref-peter2019}{}%
Peter, M. S., Durrant, S., Jessop, A., Bidgood, A., Pine, J. M., \&
Rowland, C. F. (2019). Does speed of processing or vocabulary size
predict later language growth in toddlers? \emph{Cognitive Psychology},
\emph{115}, 101238.

\leavevmode\hypertarget{ref-Roy2015}{}%
Roy, B. C., Frank, M. C., Decamp, P., Miller, M., \& Roy, D. (2015).
Predicting the birth of a spoken word. \emph{PNAS}, \emph{112}(41),
12663--12668. \url{http://doi.org/10.1073/pnas.1419773112}

\leavevmode\hypertarget{ref-TheManyBabiesConsortium2020}{}%
The ManyBabies Consortium. (2020). Quantifying sources of variability in
infancy research using the infant-directed speech preference.
\emph{Advances in Methods and Practices in Psychological Science}.
\url{http://doi.org/10.1177/2515245919900809}

\bibliographystyle{apacite}


\end{document}
